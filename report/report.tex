\documentclass[12pt]{report}
\usepackage{graphicx}
\usepackage[utf8]{vietnam}
\usepackage[left=2cm, right=2cm, top=2cm, bottom=2cm]{geometry}
\usepackage{subfiles}
\usepackage{fancyhdr}
\usepackage{hyperref}
\usepackage{etoolbox}
\usepackage{float}
\usepackage{array}
\usepackage{xcolor}
\usepackage{listings}
\usepackage{verbatim}
\usepackage{amsmath}
\usepackage{mathtools}
\usepackage{amssymb}

\newcommand{\tabitem}{~~\llap{\textbullet}~~}
\newcommand{\HRule}{\rule{\linewidth}{0.5mm}}

\DeclarePairedDelimiter{\ceil}{\lceil}{\rceil}

\usepackage[nottoc,notlof,notlot]{tocbibind} 

% Link color setup
\hypersetup{
	colorlinks = true,
	linkcolor = blue,
	citecolor = blue
}

% Change format of page
\pagestyle{fancy}
\fancyhf{}
\fancyhead{}
\fancyfoot{}
\fancyhead[L]{Hệ trợ giúp quyết định}
\fancyfoot[L]{Tạ Quang Tùng - KSTN-CNTT-K60 - MSSV: 20154280}
\fancyfoot[R]{\thepage}
\renewcommand{\headrulewidth}{1pt}
\renewcommand{\footrulewidth}{1pt}

\renewcommand{\thesection}{\arabic{section}}
\renewcommand{\thesubsection}{\thesection.\arabic{subsection}}
\renewcommand{\thesubsubsection}{\thesubsection.\arabic{subsubsection}}

% format
\usepackage{titlesec}
\usepackage{etoolbox}
\makeatletter
\patchcmd{\ttlh@hang}{\parindent\z@}{\parindent\z@\leavevmode}{}{}
\patchcmd{\ttlh@hang}{\noindent}{}{}{}
\makeatother

\titleformat{\subsection}
{\normalfont\large\bfseries}{\thesubsection}{1em}{}
\titleformat{\subsubsection}
{\normalfont\large\sffamily\bfseries}{\thesubsubsection}{1em}{}

% tab command
\newcommand\tab[1][1cm]{\hspace*{#1}}

\begin{document}

\subfile{sections/title-page.tex}

\pagenumbering{gobble}
\tableofcontents 
\newpage

\pagenumbering{arabic}
\newpage
\setcounter{page}{1}

{\small \textbf{Tóm tắt} - \textit{
Nguyên lý mở rộng của Zadeh là một trong những 
kỹ thuật cổ điển nhất trong lý thuyết tập mờ. Nó là một công cụ 
dùng để mở rộng một cách tự nhiên những hàm liên tục giá trị 
thực thành những hàm mờ (những hàm mà tham số đầu vào là tập mờ). 
Về mặt lý thuyết thì đây là một công cứ toán học rất mạnh mẽ và
được sử dụng rất nhiều trong lý thuyết. Tuy nhiên, việc tính toán 
chính xác hay thậm chí tính toán xấp xỉ đều thường rất khó 
khăn trong thực tế. Do đó, đã có rất nhiều cách tiếp cận 
đã được đề xuất để tính toán nguyên lý mở rộng. 
Trong bài báo cáo này, ta sẽ xem xét đến một thuật toán 
mới mẻ cho việc tính toán nguyên lý mở rộng Zadeh 
trên các hàm piecewise linear liên lục. Một trong những 
ưu điểm của phương pháp này là có thể áp dụng cho các hàm 
hoặc tập mờ mà không liên tục từ đầu vào hoặc trong quá trình tính toán 
sử dụng nguyên lý mở rộng. 
}
}

\section{Giới thiệu}
\subfile{sections/introduction.tex}

\section{Khái niệm cơ bản}
\subfile{sections/basic.tex}

\section{Thuật toán}
\subfile{sections/algorithm.tex}

\section{Cài đặt thử nghiệm}
\subfile{sections/implementation.tex}

\section{Kết luận}
\subfile{sections/conclusion.tex}

\bibliography{report}{}
\bibliographystyle{plain}

\end{document}
