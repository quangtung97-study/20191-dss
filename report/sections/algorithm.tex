\documentclass[../report.tex]{subfiles}

\begin{document}
Phần đầu của chương này sẽ là những giải thích cho thuật toán sẽ được 
đề cập. Người đọc lưu ý rằng thuật toán đề xuất chỉ áp dụng cho 
cho các hàm \textit{piecewise linear} liên tục và chúng ta giả thiết 
$X = [0, 1]$, mặc dù thuật toán có thể áp dụng cho bất kì khoảng đóng nào 
trong tập $\mathbb{R}$.

Mục đính của thuật toán là tính toán quỹ đạo của hệ động học mờ 
$(\mathbb{F}^1(X), z_f)$, là mở rộng của hệ động học rời rạc $(X, f)$.
Các tập mờ $A$ được sử dụng trong thuật toán cũng là các hàm 
\textit{piecewise linear} $A: X \rightarrow [0, 1]$ (không nhất 
thiết là liên tục). 

Ý tưởng chính của thuật toán này là tính toán nguyên lý mở rộng Zadeh 
trên các khoảng nhỏ mà trên đó đồng thời cả $f$ và $A$ đều là tuyến tính. 
Bằng cách này, chúng ta không phải xấp xỉ quỹ đạo của $A$ mà có thể tính
toán chính xác. 

Bước 1 mô tả làm sao để phân rã $X = [0, 1]$ ra thành hữu hạn các khoảng 
nhỏ để có thể tính toán trực tiếp. Kết quả của bước này là chúng ta 
thu được một phân rã hữu hạn $J$ của $[0, 1]$ mà sao cho với mọi 
$K \in J$ không tồn tại khoảng lớn hơn chứa $K$ mà cả $A$ và $f$ trong 
khoảng đó đều tuyến tính. Điều kiện của $J$ này giúp chúng ta phải 
tính toán trên ít các đoạn nhất có thể. 

Trong bước 2, chúng ta sử dụng tính chất rằng nếu cả $A$ và $f$ đều 
là ánh xạ tuyến tính trên khoảng $K$ thì khi áp dụng nguyên lý mở rộng 
Zadeh cho $f$ và $A|_{K}$ thì kết quả cũng tuyến tính trên $f(K)$.
Với mỗi khoảng $K \in J$, áp dụng các tính trên ta 
thu được tập các đoạn thẳng. Nếu lấy supremum của tập các đoạn thẳng 
đó ta sẽ thu được ảnh của $z_f(A)$ của tập mờ $A$. 

Bước 3 là lặp lại bước 1 và 2 cho đến khi nào đạt tới số 
vòng lặp $M$ mong muốn. \\[2mm]

INPUT:
\begin{itemize}
\item Một hàm \textit{piecewise linear} liên tục
$f: [0, 1] \rightarrow [0, 1]$
được cho bởi các điểm $(c_i, s_i)$. Hay một cách chính xác: 
$$
f(x) =
\begin{cases} 
s_1 + (s_2 - s_1) \cdot \frac{x - c_1}{c_2 - c_1} & c_1 \le x \le c_2 \\
s_2 + (s_3 - s_2) \cdot \frac{x - c_2}{c_3 - c_2} & c_2 \le x \le c_3 \\
\vdots&\\
s_{l - 1} + (s_l - s_{l - 1}) \cdot \frac{x - c_{l - 1}}{c_l - c_{l - 1}} & 
c_{l - 1} \le x \le c_l
\end{cases}
$$

\item Một số $M$ là số bước lặp. 
\item Khởi tạo $k = 1$.
\end{itemize}

BƯỚC 1:
\begin{itemize}
\item Tạo một tập $J$ bao gồm $s$ đoạn thẳng $L_i$ sinh bởi các đoạn 
    thẳng từ tập A bởi thuật giải đệ quy sau: Với mỗi $c_j \in (a_i, a_i')$
    từ đoạn $L_i' := ((a_i, b_i), (a_i', b_i')) \in J$ và m là số lượng
    phần tử trong $J$. Thì J sẽ được thay thế $L_i'$ bằng hai phần 
    tử $L_i=((a_i, b_i), (c_j, A(c_j)))$ và 
    $L_{m + 1} = ((c_j, A(c_j)),(a_i', b_i'))$
\end{itemize}

BƯỚC 2:
\begin{itemize}
\item Với mỗi $i = 1, 2, \cdots, s$ và mỗi đoạn thẳng 
    $L_i = ((a_i, b_i), (a_i', b_i'))$ từ $J$, 
    ta sẽ tính ảnh của $z_f(L_i)$ bằng nguyên lý mở rộng Zadeh của 
    hàm $f$ giới hạn trên đoạn $[a_i, a_i']$. Một cách chính xác: 
    $z_f(L_i) = ((f(a_i), b_i), (f(a_i'), b_i'))$.
\item Lấy supremum của các đoạn thẳng thu được, kết quả là tập mờ 
    $z_f^k(A)$.
\end{itemize}

BƯỚC 3:
\begin{itemize}
\item Nếu $k = M$, thuật toán kết thúc. 
\item Nếu $k < M$, thay thế $k = k + 1$ và lặp lại từ BƯỚC 1.
\end{itemize}

OUTPUT:
\begin{itemize}
\item Tập các tập mờ $z_f(A),\cdots,z_f^M(A)$
\end{itemize}


\end{document}
