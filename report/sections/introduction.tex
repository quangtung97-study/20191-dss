\documentclass[../report.tex]{subfiles}

\begin{document}
Nguyên lý mở rộng của Zadeh (Zadeh's extension principle) chiếm một phần 
quan trọng trong toán học mờ bởi vì khả năng dễ dàng mở rộng rất nhiều những 
phép toán cổ điển (không phải mờ) sang thàng các phép toán mờ tương ứng \cite{Zadeh-extension}.

Nguyên lý mở rộng này cũng có thể được sử dụng để mở rộng hàm giá trị một biến,
bằng việc sử dụng công thức sau:
\begin{equation}
(z_f(A))(x) = \sup_{y \in f^{-1}(x)} \{ A(y) \}
\label{eq:1}
\end{equation}
mỗi ánh xạ $f: X \rightarrow Y$ định nghĩa duy nhất một ánh xạ:
$z_f: \mathbb{F}(X) \rightarrow \mathbb{F}(Y)$, trong đó $\mathbb{F}(X)$
(tương ứng $\mathbb{F}(Y)$) là một lớp tương ứng các tập mờ 
định nghĩ trên $X$.

Biểu thức \eqref{eq:1} có thể được sử dụng để mở rộng
lý thuyết các hệ động học. Cặp $(X, f)$ với X là không gian topo (topological space) 
và $f: X \rightarrow X$ là ánh xạ liên tục được gọi là một hệ động học rời rạc
(\textit{discrete dynamical system}). Kloeden trong bài báo \cite{Kloeden}


\end{document}
