\documentclass[../report.tex]{subfiles}

\begin{document}
Nguyên lý mở rộng của Zadeh (Zadeh's extension principle) chiếm một phần 
quan trọng trong toán học mờ bởi vì khả năng dễ dàng mở rộng rất nhiều những 
phép toán cổ điển (không phải mờ) sang thàng các phép toán mờ tương ứng \cite{Zadeh-extension}.

Nguyên lý mở rộng này cũng có thể được sử dụng để mở rộng hàm giá trị một biến,
bằng việc sử dụng công thức sau:
\begin{equation}
(z_f(A))(x) = \sup_{y \in f^{-1}(x)} \{ A(y) \}
\label{eq:1}
\end{equation}
mỗi ánh xạ $f: X \rightarrow Y$ định nghĩa duy nhất một ánh xạ:
$z_f: \mathbb{F}(X) \rightarrow \mathbb{F}(Y)$, trong đó $\mathbb{F}(X)$
(tương ứng $\mathbb{F}(Y)$) là một lớp tương ứng các tập mờ 
định nghĩ trên $X$.

Biểu thức \eqref{eq:1} có thể được sử dụng để mở rộng
lý thuyết các hệ động học. Cặp $(X, f)$ với X là không gian topo (topological space) 
và $f: X \rightarrow X$ là ánh xạ liên tục được gọi là một hệ động học rời rạc
(\textit{discrete dynamical system}).
Kloeden trong bài báo \cite{Kloeden} đã đề xuất một mô hình toán học 
kết nối giữa lý thuyết của hệ động học (không mờ) sang lý thuyết hệ động học mờ. 
Bằng việc sử dụng mô hình này và phương trình \eqref{eq:1}, cho bất kì một hệ động
học $(X, f)$, ta có thể suy ra hệ động học mờ $(\mathbb{F}(X), z_f)$, từ đó ta 
có thể nghiên cứu các tính chất của các hệ động lực mờ từ những tính chất đã được 
nghiên cứu hàng thập kỷ cho các hệ động học cổ điển. Đã có nhiều bài báo khoa học
nghiên cứu về topic này như \cite{topological-entropy}, \cite{diffusion-equation},
\cite{Kloeden} \cite{Kupka}.

Tuy nhiên, vấn đề thường xuyên xảy ra khi mà cần tính toán phương trình \eqref{eq:1}
các bài toán ứng dụng. Nhìn chung các bài toán liên quan đến nguyên lý mở rộng 
Zadeh đều có thể là những bài toán khó \cite{Zadeh-extension}.
Sự khó khăn đến từ việc tính hàm ngược $f^{-1}$
tại điểm bất kì. Do đó, rất nhiều nhóm nghiên cứu đã đề xuất nhiều cách tiếp cận 
khác nhau để tìm giá trị xấp xỉ của ảnh của tập mờ khi áp dụng nguyên lý mở rộng của Zadeh.

Trong báo cáo này ta chỉ chú ý tới những tập mờ đặc biệt có tên:
\textit{piecewise linear fuzzy set}.
Với các loại tập mờ này, giải thuật được đề xuất có thể được sử dụng để tính 
nguyên lý mở rộng Zadeh. Khác với các cách tiếp cận đã có, cách tiếp cận này 
không phải là phương pháp xấp xỉ mà là phương pháp tính toán chính xác áp dụng 
cho hệ động lực mờ.  
Mặc dù ta không xét đến các ánh xạ liên tục bất kì mà chỉ cho các ánh xạ
piecewise linear. Tuy nhiên tập các ánh xạ piecewise linear này là
\textit{dense} trong tập các ánh xạ liên tục nên có thể xấp xỉ một 
ánh xạ liên lục bất kì với sai số nhỏ tùy ý.
Đồng thời cách tiếp cận này có thể áp dụng cho các ánh xạ không liên tục.

\end{document}
